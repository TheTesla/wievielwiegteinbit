\documentclass[aspectratio=169]{beamer}
%\documentclass{easytikz}


\usepackage{default}
\usepackage{tikz}
\usetikzlibrary{positioning,automata,calc}
\usepackage{multicol}
\usepackage{svg}




\usepackage[utf8]{inputenc}
\usepackage[ngerman]{babel}
\usepackage[T1]{fontenc}
\usepackage{minibox}
\usepackage{setspace}
\usepackage{tabularx}
\usepackage{units}
\usepackage{ulem}


\usetikzlibrary{shapes.geometric, arrows}
\usetikzlibrary{positioning}

\newenvironment{conditions}
{\par\vspace{\abovedisplayskip}\noindent\begin{tabular}{>{$}l<{$} @{${}\hspace{0.1cm}.\hspace{0.03cm}.\hspace{0.25cm}{}$} l}}
	{\end{tabular}\par\vspace{\belowdisplayskip}}


\def\doubleunderline#1{\underline{\underline{#1}}}

\title{Wie viel wiegt ein Bit?}
%\subtitle{}
\author{Stefan Helmert}
\institute{entroserv.de}
\date{03/2020}

\logo{\Large\texttt{\minibox{entroserv\vspace{-.27cm}\\\hspace{0.63cm} course}}}

\tikzset{
  se/.style={rectangle, rounded corners, minimum width=3cm, minimum height=1cm,text centered, draw=black, fill=red!30},
  sed/.style={rectangle, rounded corners, minimum width=3cm, minimum height=1cm,text centered, draw=black, fill=black!30},
  io/.style={trapezium, trapezium left angle=70, trapezium right angle=110, minimum width=3cm, minimum height=1cm, text centered, draw=black, fill=blue!30},
  op/.style={rectangle, minimum width=3cm, minimum height=1cm, text centered, draw=black, fill=orange!30},
  cn/.style={diamond, minimum width=3cm, minimum height=1cm, text centered, draw=black, fill=green!30},
  sc/.style={circle, minimum width=2cm, minimum height=1cm, text centered, draw=black, fill=blue!30},
  node distance=7mm
}

\begin{document}
	
\frame{\titlepage}

\section{Was ist Information?}
\begin{frame}[fragile]{\insertsection}{\insertsubsection}
\begin{tikzpicture}[scale=1.5]
\draw (0,0) node[fill=white] {Information} -- (-3,3) node[fill=white] {Energie} -- (3,3) node[fill=white] {Materie} -- cycle;
\end{tikzpicture}
\end{frame}

\section{Was ist ein Signal?}
\begin{frame}[fragile]{\insertsection}{\insertsubsection}
Signal ist:
\begin{itemize}
	\item Träger der Information
	\item physikalische Größe
\end{itemize}
\vspace{0.8cm}
\begin{tikzpicture}[scale=1.0,line width=0.3mm]
\draw (0,0) --++ (2,0) --++ (0,2.5) --++ (-0.5,0.5) --++ (-1.5,0) -- cycle;
\draw (2,2.5) --++ (-0.5,0) --++ (0,0.5);
\node at(0.5,2.5) {\scalebox{3}{A}};
\end{tikzpicture}
\hspace{1cm}
\begin{tikzpicture}[scale=1.5,line width=0.3mm]
\draw (0,0) --++ (0,1) --++ (0.4,0) --++ (0,-1) -- cycle;
\draw (0.4,0.25) --++(0.6,-0.5) --++ (0,1.5) --++ (-0.6,-0.5);
\draw (0,0.5) ++ (20:1.8) arc (20:-20:1.8);
\draw (0,0.5) ++ (20:2.2) arc (20:-20:2.2);
\draw (0,0.5) ++ (20:2.6) arc (20:-20:2.6);
\end{tikzpicture}
\end{frame}


\section{Kanalkapazität}
\centering\begin{frame}[fragile]{\insertsection}{\insertsubsection}
\begin{equation}
	\underbrace{C}_{\unit{\frac{bit}{s}}} = \underbrace{B}_{\unit{\frac{symbols}{s}}} \cdot \underbrace{\log_2\left(1 + \frac{S}{N}\right)}_{\unit{\frac{bit}{symbol}}}
	\label{eq:kanalkap}
\end{equation}
\begin{conditions}
	C & Kanalkapazität \\
	B & Bandbreite \\   
	S & Signalleistung \\
	N & Rauschleistung \\
	\frac{S}{N} & Signal-Rausch-Verhältnis
\end{conditions}
\end{frame}

\section{Informationsgehalt}
\centering\begin{frame}[fragile]{\insertsection}{\insertsubsection}
\begin{equation}
	\underbrace{I}_{\unit{bit}} = \underbrace{C}_{\unit{\frac{bit}{s}}} \cdot \underbrace{t}_{\unit{s}}
	\label{eq:infom}
\end{equation}
\begin{conditions}
	I & Informationsgehalt \\
	C & Kanalkapazität \\
	t & Übertragungszeit  
\end{conditions}
\end{frame}


\centering\begin{frame}[fragile]{\insertsection}{\insertsubsection}
\begin{flalign}
(\ref{eq:kanalkap}\rightarrow\ref{eq:infom})&& \underbrace{I}_{\unit{bit}} &= \underbrace{B \cdot t}_{\unit{symbols}} \cdot \underbrace{\log_2\left(1 + \frac{S}{N}\right)}_{\unit{\frac{bit}{symbol}}} &&
\label{eq:informsnr}
\end{flalign}
\begin{conditions}
	I & Informationsgehalt \\
	B & Bandbreite \\   
	t & Übertragungszeit \\
	S & Signalleistung \\
	N & Rauschleistung \\
	\frac{S}{N} & Signal-Rausch-Verhältnis
\end{conditions}
\end{frame}

\section{Symbolanzahl}
\centering\begin{frame}[fragile]{\insertsection}{\insertsubsection}
\begin{flalign}
(\ref{eq:informsnr}\ \text{umgestellt}) && \underbrace{B \cdot t}_{\unit{symbols}} &= \frac{I}{\log_2\left(1 + \frac{S}{N}\right)} &&
\label{eq:symbolssnr}
\end{flalign}
\begin{conditions}
	B & Bandbreite \\   
	t & Übertragungszeit \\
	I & Informationsgehalt \\
	S & Signalleistung \\
	N & Rauschleistung \\
	\frac{S}{N} & Signal-Rausch-Verhältnis
\end{conditions}
\end{frame}

\section{Thermisches Rauschen}
\centering\begin{frame}[fragile]{\insertsection}{\insertsubsection}
\begin{flalign}
\underbrace{N}_{\unit{W}} = \underbrace{k_B}_{\frac{\unit{Ws}}{\unit{symbol\cdot K}}} \cdot \underbrace{T}_{\unit{K}} \cdot \underbrace{B}_{\unit{\frac{symbols}{s}}}
\label{eq:thermalnoise}
\end{flalign}
\begin{conditions}
	N & Rauschleistung \\
	k_B & Boltzmann-Konstante \\
	T & Temperatur \\
	B & Bandbreite \\   
\end{conditions}
\end{frame}

\section{Energie und Leistung}
\centering\begin{frame}[fragile]{\insertsection}{\insertsubsection}
\begin{equation}
\begin{array}{c @{{}={}} c @{{}\cdot{}} c}
 \underbrace{E}_{\unit{Ws}} &\underbrace{P}_{\unit{W}} & \underbrace{t}_{\unit{s}}\\
  E_N & N & t  \\
  E_S & S & t
\end{array}
\label{eq:power}
\end{equation}
\begin{conditions}
	E & Energie \\
	P & Leistung \\
	t & Zeit \\
	E_N & Rauschenergie \\
	N & Rauschleistung \\
	E_S & Signalenergie \\
	S & Signalleistung \\
\end{conditions}
\end{frame}

\section{Rauschenergie}
\centering\begin{frame}[fragile]{\insertsection}{\insertsubsection}
\begin{flalign}
(\ref{eq:thermalnoise} \rightarrow \ref{eq:power}) &&
\underbrace{E_N}_{\unit{Ws}} &=  \underbrace{k_B}_{\frac{\unit{Ws}}{\unit{symbol\cdot K}}} \cdot \underbrace{T}_{\unit{K}} \cdot \underbrace{B}_{\unit{\frac{symbols}{s}}} \cdot \underbrace{t}_{\unit{s}} &&
\label{eq:noiseenergy}
\end{flalign}
\begin{conditions}
	E_N & Rauschenergie \\
	k_B & Boltzmann-Konstante \\
	T & Temperatur \\
	B & Bandbreite \\   
	t & Übertragungszeit
\end{conditions}
\end{frame}


\centering\begin{frame}[fragile]{\insertsection}{\insertsubsection}
\begin{flalign}
(\ref{eq:symbolssnr} \rightarrow \ref{eq:noiseenergy}) &&
\underbrace{E_N}_{\unit{Ws}} &= \underbrace{k_B}_{\frac{\unit{Ws}}{\unit{symbol\cdot K}}} \cdot \underbrace{T}_{\unit{K}} \cdot \underbrace{\frac{I}{\log_2\left(1 + \frac{S}{N}\right)}}_{\unit{symbols}} &&
\label{eq:noiseenergysnr}
\end{flalign}
\begin{conditions}
	E_N & Rauschenergie \\
	k_B & Boltzmann-Konstante \\
	T & Temperatur \\
	I & Informationsgehalt \\
	S & Signalleistung \\
	N & Rauschleistung \\ 
	\frac{S}{N} & Signal-Rausch-Verhältnis
\end{conditions}
\end{frame}

\section{Erforderliche Signalenergie}
\centering\begin{frame}[fragile]{\insertsection}{\insertsubsection}
\begin{equation}
\underbrace{E_S}_{\unit{Ws}} = \underbrace{\frac{S}{N}}_1 \cdot \underbrace{E_N}_{\unit{Ws}}
\label{eq:neededsignal}
\end{equation}
\begin{conditions}
	E_S & Signalenergie \\
	S & Signalleistung \\
	N & Rauschleistung \\ 
	\frac{S}{N} & Signal-Rausch-Verhältnis \\
	E_N & Rauschenergie \\
\end{conditions}
\end{frame}

\centering\begin{frame}[fragile]{\insertsection}{\insertsubsection}
\begin{flalign}
(\ref{eq:noiseenergysnr} \rightarrow \ref{eq:neededsignal}) && \underbrace{E_S}_{\unit{Ws}} &= \underbrace{\frac{S}{N}}_1 \cdot  \underbrace{k_B}_{\frac{\unit{Ws}}{\unit{symbol\cdot K}}} \cdot \underbrace{T}_{\unit{K}} \cdot \underbrace{\frac{I}{\log_2\left(1 + \frac{S}{N}\right)}}_{\unit{symbols}} &&
\label{eq:neededsignalsnr}
\end{flalign}
\begin{conditions}
	E_S & Signalenergie \\
	S & Signalleistung \\
	N & Rauschleistung \\ 
	\frac{S}{N} & Signal-Rausch-Verhältnis \\
	k_B & Boltzmann-Konstante \\
	T & Temperatur \\
	I & Informationsgehalt \\
\end{conditions}
\end{frame}

\section{Erforderliche Signalenergie nach SNR}
\centering\begin{frame}[fragile]{\insertsection}{\insertsubsection}
\begin{alignat*}{3}
\frac{S}{N} & = 1 & \rightarrow E_S & = & 1 & \cdot k_B \cdot T \cdot I \\
\frac{S}{N} & = 3 & \rightarrow E_S & = & \frac{3}{2} & \cdot k_B \cdot T \cdot I \\ 
\frac{S}{N} & = 7 & \rightarrow E_S & = & \frac{7}{3} & \cdot k_B \cdot T \cdot I \\
\frac{S}{N} & = 0{,}001 & \rightarrow E_S & = & \frac{1000}{1442} & \cdot k_B \cdot T \cdot I \\ 
\frac{S}{N} & = 0{,}000001 & \rightarrow E_S & = & \frac{1000}{1443} & \cdot k_B \cdot T \cdot I 
\end{alignat*}
\end{frame}

\section{Energie für ein Bit}
\centering\begin{frame}[fragile]{\insertsection}{\insertsubsection}
\begin{alignat}{3}
k_B & = & 1&{,}3806504 \cdot 10^{-23}\ \frac{\text{Ws}}{\text{K}} \\
T & = & 300 &\ \unit{K}\\
I & = & 1 &\ \unit{bit}
\end{alignat}
\begin{conditions}
	k_B & Boltzmann-Konstante \\
	T & Temperatur ($26^\circ$C Umgebungstemperatur)\\
	I & Informationsgehalt \\
\end{conditions}
\end{frame}

\centering\begin{frame}[fragile]{\insertsection}{\insertsubsection}
\begin{alignat}{3}
\frac{S}{N} = 0{,}000001 \rightarrow E_S & = \frac{1000}{1443} \cdot 1{,}3806504 \cdot 10^{-23} \frac{\text{Ws}}{\text{K}} \cdot 300 \text{K} \cdot 1 \\
\makebox[0pt][l]{\uuline{\phantom{$ E_S = 6 \cdot 10^{-21} \text{Ws}$}}}
 E_S & = 6 \cdot 10^{-21} \text{Ws}
\end{alignat}
\end{frame}

\section{Masse}
\centering\begin{frame}[fragile]{\insertsection}{\insertsubsection}
\begin{equation}
E = mc^2
\label{eq:relativ}
\end{equation}
\begin{conditions}
	E & Energie \\
	m & Masse \\
	c & Lichtgeschwindigkeit \\
\end{conditions}
\end{frame}

\centering\begin{frame}[fragile]{\insertsection}{\insertsubsection}
\begin{flalign}
(\ref{eq:relativ}\ \text{umgestellt}) && m &= \frac{E}{c^2} &&
\end{flalign}
\begin{conditions}
	m & Masse \\
	E & Energie \\
	c & Lichtgeschwindigkeit \\
\end{conditions}
\end{frame}

\centering\begin{frame}[fragile]{\insertsection}{\insertsubsection}
\begin{alignat}{3}
c & = & 3 & \cdot 10^{8}\ \unit{\frac{m}{s^2}}  \\
\end{alignat}
\begin{conditions}
	c & Lichtgeschwindigkeit \\
\end{conditions}
\end{frame}

\centering\begin{frame}[fragile]{\insertsection}{\insertsubsection}
\begin{alignat}{2}
m_{\unit{bit}} & = \frac{6 \cdot 10^{-21} \text{Ws}}{(3\cdot 10^8 \text{m/s})^2} \\
\makebox[0pt][l]{\uuline{\phantom{$m_{\unit{bit}} = 6{,}67 \cdot 10^{-38} \text{kg}$}}}
m_{\unit{bit}} & = 6{,}67 \cdot 10^{-38} \text{kg}
\end{alignat}
\end{frame}

\end{document}
